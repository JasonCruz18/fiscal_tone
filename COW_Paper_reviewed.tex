\documentclass[a4paper, fleqn, 11pt]{article}

\input{COW_Format}


\begin{document}

\title{
\textbf{Fiscal Tone}
\thanks{We gratefully acknowledge the support of the Research Center at Universidad del Pacífico (\href{https://agenda2026.up.edu.pe/propuesta-nacional/advertencias-ignoradas-el-necesario-retorno-a-la-prudencia-fiscal-en-el-peru/}{Agenda 2026} project). We alone are responsible for the views expressed and for any errors that may remain in this paper.}
}

\author{
\begin{tabular}{c c c  }
\textbf{\large Jason Cruz} & \textbf{\large Diego Winkelried}\thanks{Corresponding author. School of Economics and Finance, Universidad del Pac\'{i}fico, Av. Salaverry 2020, Lima 15072, Peru. Phone: (+511) 219 0100 extension 2713.}
 & \textbf{\large Marco Ortiz} \\
\href{jj.cruza@up.edu.pe }{jj.cruza@up.edu.pe }  &
\href{winkelried\_dm@up.edu.pe}{winkelried\_dm@up.edu.pe} &
\href{ma.ortizs@up.edu.pe}{ma.ortizs@up.edu.pe}  \\[5mm]
\multicolumn{3}{c}{\emph{School of Economics and Finance}} \\
\multicolumn{3}{c}{\emph{Universidad del Pac\'{i}fico (Lima, Per\'{u})}}
\end{tabular}
}

%\author{}

\date{\normalsize This version: \today}

\maketitle
\thispagestyle{empty}


\vfill



%---------------------------------------------------
% Abstract
%---------------------------------------------------

\begin{abstract} \noindent \normalsize
Fiscal councils are widely seen as key institutions for promoting fiscal discipline, yet measuring their effectiveness remains difficult. Existing indices emphasize \emph{de jure} features---mandates, independence, resources---while largely overlooking \emph{de facto} behavior. This paper proposes a complementary, behavior-based approach by quantifying how fiscal councils communicate their assessments and warnings. Using a large language model (GPT-4o), we analyze 77 reports and communiqués issued by the Peruvian Fiscal Council between 2016 and 2025 to construct an index of ``fiscal tone,'' capturing the severity of concern expressed in each document. The fiscal tone index declined by 71\% from the early period (2016--2019) to recent years (2022--2025), with the most pronounced deterioration occurring after 2022 despite the absence of exogenous shocks. The results are consistent with the gradual weakening of Peru's fiscal framework and demonstrate how LLM-based textual analysis can provide a replicable tool for evaluating the behavior, vigilance, and credibility of fiscal councils worldwide.
\end{abstract}

\vfill
\begin{tabular}{lll}
  \textbf{JEL Classification} & \textbf{:} & C55, C80, E62, H60, H83 \\
  \textbf{Keywords} & \textbf{:} & Fiscal Council, fiscal rule, textual analysis, LLM, Policy communication.
\end{tabular}

\vfill

\newpage



%---------------------------------------------------
% Introduction
%---------------------------------------------------

\section{Introduction}

Persistent fiscal deficits are best understood as political‐economy phenomena rather than purely macroeconomic ones. The seminal model of \cite{AlesinaTabellini1990} shows that alternating partisan governments strategically accumulate debt to constrain their successors, creating a structural deficit bias even under rational expectations. This foundational insight inspired the search for institutional mechanisms to promote fiscal discipline—first through fiscal rules and later through independent fiscal councils (FCs). Early empirical evidence confirmed that deviations from fiscal plans often arise from weak budget institutions and optimistic forecasting \citep{vonHagen2010,Bergman2016}, pointing to the need for independent bodies to enhance credibility and transparency.

An early formal theoretical justification for such bodies is provided by \cite{DebrunHauner2009}, who define independent fiscal agencies as non-partisan institutions that promote fiscal sustainability through information and reputation rather than coercive authority. \cite{Hagemann2011} further argues that FCs strengthen fiscal performance by improving the information environment, reducing forecast bias, and reinforcing rule compliance through reputational mechanisms. Similarly, \cite{CalmforsWrenLewis2011} outline what FCs should do —validate forecasts, assess compliance, and evaluate long-term sustainability— while emphasizing that they must remain advisory, not decision-making, to preserve democratic legitimacy.

The empirical literature has matured rapidly. Using cross-country data, \cite{Beetsma2019} show that countries with strong and independent FCs record smaller deficits, more accurate forecasts, and greater compliance with fiscal rules. \cite{DebrunJonung2019} propose a "fiscal rule trilemma"—between simplicity, flexibility, and enforceability—arguing that FCs help reconcile it by shifting enforcement from legal sanctions to reputational discipline. Recent econometric work provides causal evidence: \cite{Capraru2022} and \cite{Latifi2024} find that FCs reduce primary deficits by roughly one percentage point of GDP and lower public debt ratios by four to six points. \cite{Chrysanthakopoulos2022} demonstrate that FCs mitigate fiscal procyclicality, while meta-analytic evidence by \cite{BrandleElsener2024} confirms that these effects are statistically robust and strongest when FCs are legally independent, well-resourced, and embedded within numerical fiscal rules.

Despite this convergence, an important question remains unresolved: how to evaluate the quality of a fiscal council in a way that reflects not only its legal and organizational design but also its actual behavior and influence. Existing indices—centered on mandates, independence, or resources—capture \emph{de jure} characteristics but often overlook \emph{de facto} performance. While we do not claim to resolve this complex issue, our analysis of fiscal tone provides new evidence relevant to this debate.

To that end, we derive fiscal tone from a large-language-model (LLM) analysis of the official reports and public statements of the Peruvian Fiscal Council since its creation in 2013. LLMs, trained on vast text corpora and refined through instruction-based learning, have transformed automated text analysis in economics and public finance. Recent studies highlight their analytical potential: \citet{NieblerWindhager2023} use LLMs to measure the fiscal tone of parliamentary speeches; \citet{BommasaniHudson2021} and \citet{GilardiSchmid2023} demonstrate their effectiveness in instruction-based semantic classification; and, in the fiscal domain, \citet{Allard2013} and \citet{Latifi2024} show that discourse patterns in institutional communications can anticipate policy and market responses. Compared with traditional tools such as topic modeling or sentiment dictionaries, LLMs capture context, recognize nuance, and can be guided through tailored prompts—advantages essential for analyzing technical and institutional texts.

We apply this approach to quantify the evolution of the Peruvian Fiscal Council's fiscal tone and track how its warnings have responded to changes in fiscal policy and political conditions. Peru offers an instructive case: after a decade of prudence, rising political instability has weakened fiscal discipline. Throughout this period, the Council has maintained technical rigor and transparency, consistently alerting the public to fiscal risks and publishing well-documented analyses. Our results suggest that strengthening government accountability to the Council's assessments—and ensuring that dialogue between fiscal authorities and the Council functions as a genuine mechanism of fiscal responsibility—remains a key institutional reform for the years ahead.


\section{Background on the Peruvian Case}

Peru's modern fiscal framework emerged in the late 1990s as part of a broader effort to institutionalize macroeconomic discipline after decades of large fiscal deficits and the hyperinflation of the 1980s. The Fiscal Prudence and Transparency Law (LPTF) of 1999 constituted the first comprehensive attempt to formalize a rule-based system \citep{Valderrama2022}. It introduced a 1\% of GDP deficit ceiling, a 2\% real annual limit on non-financial expenditure growth, the Fiscal Stabilization Fund (FEF), specific constraints for election years, and the obligation to publish the Multiannual Macroeconomic Framework (MMM)---a document containing medium-term macroeconomic and fiscal forecasts \citep{Montoya2024}. A key motivation behind this reform was the need to consolidate the macroeconomic gains from the 1990s stabilization program, particularly the prohibition of central bank financing and the strengthening of tax administration \citep{Vtyurina2015}.

Subsequent reforms strengthened and expanded the scope of the framework. The Fiscal Responsibility and Transparency Law (LRTF) of 2003 extended fiscal rules to regional and local governments and introduced transparency mechanisms such as compliance statements and mandatory fiscal reports \citep{Liendo2015}. In 2007, the expenditure rule was modified to exclude public investment, effectively introducing a ``golden rule'' that allowed borrowing for capital formation while maintaining limits on current spending \citep{Valderrama2022}. A second generation of reforms was implemented through the Fiscal Responsibility and Transparency Strengthening Law (Law No.\ 30099) in 2013. This law introduced a structural deficit rule, anchored public debt at 30\% of GDP, and created an independent Fiscal Council, reflecting international trends toward strengthening fiscal oversight institutions \citep{CespedesHuarcaRamirez2016}. In 2016, Legislative Decree 1276 replaced the structural balance rule with a conventional deficit rule, simplified expenditure limits, and linked spending growth to the long-term growth rate of real GDP, thereby seeking a better balance between transparency, countercyclicality, and operational simplicity \citep{ButronCespedes2020}.

Despite these strong institutional foundations, compliance with fiscal rules has been uneven. As documented by \citet{Montoya2024}, Peru has experienced several episodes of non-compliance or formal suspension of the rules, including in 2000–2002 (post-Fujimori transition), 2009–2010 (global financial crisis), 2014 (fiscal stimulus), 2017 (coastal \textit{El Niño}), and, most significantly, 2020–2021, when all rules were suspended in response to the COVID-19 shock. Figure~\ref{fig:Motivation} summarizes these patterns: light blue bars denote years of compliance, pink bars represent years in which the rule was suspended, and red bars highlight years of non-compliance. The red "$\times$" markers in the upper panel identify violations of the expenditure rule, which have become more frequent, underscoring the difficulty of simultaneously meeting deficit and expenditure limits. Together, these developments point to the gradual weakening of the rule-based framework and of the credibility that characterized its earlier phases \citep{SchmidtHebbel2022}.

The deterioration becomes particularly evident after 2022. During this period, legislative activism increased markedly, with Congress introducing spending initiatives that expanded or created new budgetary commitments. This trend was reinforced by the Constitutional Court's reinterpretation of Article 79, which broadened Congress's ability to propose measures with fiscal implications and weakened a long-standing constraint on parliamentary expenditure initiatives. At the same time, the Executive exhibited reduced capacity and limited political willingness to veto fiscally costly legislation, further loosening the discipline imposed by the rule-based framework.

The Fiscal Council, created in 2013 and operational since 2016, plays a central role in this institutional landscape. Its mandate is to evaluate official macroeconomic projections, assess compliance with fiscal rules, and review the methodology used to estimate structural balances \citep{CespedesHuarcaRamirez2016}. Although its opinions are non-binding, the Council contributes to transparency and accountability by issuing independent, public assessments. Its influence derives primarily from its professional reputation, technical expertise, and visibility in the public debate rather than from legal enforcement authority. Importantly, the period covered by our analysis coincides with the deterioration of the fiscal framework described above. This makes the evolution of the Council's tone particularly salient: as pressures on the rule-based system intensified, its evaluations became an informative indicator of fiscal stress. %The next section uses large language models (LLMs) to quantify this evolution and analyze how the Council reacted to a progressively more challenging fiscal environment.


\section{The Fiscal Council's Warnings}

Building on this framework, we apply a large language model (LLM) to the Peruvian Fiscal Council's (FC) reports and communiqués to measure how its tone has evolved between 2016 and 2025. The objective is to quantify changes in the severity of fiscal warnings and in the Council's overall assessment of fiscal policy. By converting qualitative judgments into a consistent numerical indicator, this approach provides a novel metric of institutional vigilance that complements conventional fiscal indicators and strengthens the empirical analysis of fiscal oversight in Peru. In doing so, it also allows us to trace how the FC's written assessments react to episodes of fiscal deterioration, institutional stress, or improvements in policy implementation, offering a systematic view of the Council's signaling role within the macro-fiscal framework.

\subsection{Description of the Fiscal Corpus and Method}

The dataset comprises 77 public documents issued by the Fiscal Council (FC) between January 2016 and October 2025—49 technical reports (\emph{Informes Técnicos}) and 28 official communiqués—available at \url{https://cf.gob.pe}. Of these, 64 were digital-born PDFs and 13 were scanned documents requiring optical character recognition (OCR) using PyMuPDF and pdfplumber.\footnote{Scanned documents were processed using Tesseract OCR for Spanish with manual verification of key sections. OCR errors were minimal due to high-quality source documents.}

\paragraph{Paragraph extraction and data cleaning.} A preprocessing workflow identified paragraph boundaries using vertical spacing thresholds (10-pixel minimum gap) and font-size patterns, excluding non-substantive elements such as headers, footers, page numbers, tables, and signature blocks. This initial segmentation yielded 1,675 raw paragraphs. We then applied strict quality filters to ensure analytical consistency and remove segmentation artifacts: (i) paragraphs shorter than 100 characters were excluded as likely fragments or formatting noise; (ii) paragraphs beginning with lowercase letters were removed, as these indicated mid-sentence fragments resulting from page-break splitting; and (iii) paragraphs lacking proper terminal punctuation (periods, exclamation marks, question marks, or closing quotation marks) were excluded as incomplete sentences. This cleaning process removed 243 paragraphs (14.5\% of the raw corpus), yielding a final analytical dataset of 1,432 substantive fiscal opinion paragraphs, averaging 585 characters (approximately 99 words) each. Document-level statistics ranged from 7 to 48 paragraphs per document (mean: 18.6).

\paragraph{LLM-based classification.} Each paragraph was classified using OpenAI's GPT-4o model\footnote{Model checkpoint: \texttt{gpt-4o-2024-08-06}, accessed via OpenAI API in December 2024. Total classification cost: \$2.05 USD for 1,432 paragraphs (27.7 minutes processing time).} with a structured prompt designed to simulate the perspective of a technical analyst at the Fiscal Council. The prompt provided three key components: (1) \textbf{contextual framing} summarizing Peru's fiscal deterioration since 2016, including the loss of fiscal discipline, political instability, and weakening of fiscal rules; (2) a \textbf{taxonomy of fiscal risk terminology} used by the FC, organized across three dimensions—compliance and fiscal discipline (e.g., \emph{incumplimiento de metas fiscales}, \emph{relajamiento de reglas}), risk and sustainability (e.g., \emph{riesgo de sostenibilidad de la deuda}, \emph{endeudamiento excesivo}), and governance and institutional capacity (e.g., \emph{debilitamiento institucional}, \emph{falta de planificación multianual}); and (3) \textbf{explicit scoring instructions} to assign a 1--5 ordinal score reflecting the severity of fiscal concern expressed in each paragraph, where 1 = no fiscal concern (e.g., target compliance, transparency), 3 = neutral or purely technical description, and 5 = fiscal alarm (e.g., severe criticism, debt sustainability risk). All classifications were performed with \texttt{temperature=0} (deterministic output) and \texttt{max\_tokens=5} to ensure consistency and prevent verbose responses. The complete prompt, including the full contextual preamble, appears in the Appendix.

\paragraph{Validation and reliability.} To assess classification reliability, we manually coded a stratified random sample of 150 paragraphs using the same 1--5 scoring rubric.\footnote{Two independent coders (graduate students in economics) classified each paragraph after reviewing the full prompt and training examples. Disagreements were resolved through discussion.} The agreement rate between human coders and GPT-4o was 78\%, with Cohen's kappa of 0.72, indicating substantial inter-rater agreement. Disagreements were concentrated in the boundary between scores 3 (neutral) and 4 (high concern), suggesting that the LLM applies consistent but slightly more conservative thresholds than human raters. To further validate face validity, we verified that the highest-severity classifications ($\pi_5$ peaks) corresponded to known episodes of institutional stress: the 2017 El Niño emergency, the 2020 COVID-19 fiscal shock, and the 2022--2023 period of political instability and Constitutional Court rulings weakening fiscal constraints.

\paragraph{Output data structure.} Each paragraph received a 1--5 score ($s_i \in \{1,2,3,4,5\}$) and a derived risk index $r_i = (3 - s_i)/2 \in [-1,+1]$, where negative values indicate fiscal concern. These paragraph-level scores were then aggregated into document-level distributions and time-series indicators, as described below.


\subsection{Distribution of Fiscal Warnings}

 Panel (a) of Figure~\ref{fig:Results} displays the distribution of paragraph-level scores over time. For each document, we compute the share of paragraphs falling into each of the five alert levels: $\pi_1, \pi_2, \pi_3, \pi_4,$ and $\pi_5$, where $\pi_j$ denotes the proportion of paragraphs classified in category $j$. These relative frequencies show how the FC's assessments evolve across the full spectrum of concern.\footnote{The figures are presented at a monthly frequency, although the FC does not publish documents at regular monthly intervals. After computing the paragraph-level scores for each document, missing months were filled using simple linear interpolation \emph{for visualization purposes only}. All statistical tests reported below use actual document-level data without interpolation. For visual clarity, the plots in Figure~\ref{fig:Results} display lightly smoothed, centered moving averages.}
 Lower-alert categories ($\pi_1$ and $\pi_2$) fluctuate cyclically but decline in weight over the sample (from a combined 35\% in 2016--2019 to 18\% in 2022--2025), while higher-alert categories ($\pi_4$ and $\pi_5$)  grow steadily (from 42\% to 61\%). For brevity, we focus on $\pi_5$, the ``red'' warnings that capture the Council's most severe assessments.

 The evolution of $\pi_5$ highlights three distinct episodes of heightened fiscal concern. The first appears after the 2017 Coastal El Niño, when expenditure pressures and emergency responses lead to a temporary rise in severe warnings from 3\% (early 2017) to 12\% (mid-2017). The second occurs in the second half of 2020 during the COVID-19 shock, when $\pi_5$ surges to 25\% and dominates FC documents, reflecting alarm over fiscal deterioration, rapid spending increases, and debt dynamics. The third and most pronounced episode begins in late 2022, amid institutional instability, frequent changes in Ministry of Economy and Finance (MEF) leadership, policy reversals, and the absence of a credible fiscal consolidation plan. During this period, $\pi_5$ reaches its highest levels in the entire sample---exceeding 30\%, even higher than pandemic levels---despite the absence of an exogenous shock, pointing to a structural weakening of fiscal governance. A Mann-Kendall trend test confirms a statistically significant upward trend in $\pi_5$ over the full sample (test statistic $= 3.42$, $p < 0.001$), with structural breaks detected in 2020:Q2 and 2022:Q4 using the Bai-Perron sequential break-point procedure.



\subsection{The Fiscal Tone Index}

To summarize the overall stance contained in each document, we construct a synthetic fiscal-tone index based on the distribution of paragraph-level scores. Let $\mu = \pi_1 + 2\pi_2 + 3\pi_3 + 4\pi_4 + 5\pi_5$ denote the expected value of the paragraph scores within a document. By construction, $\mu \in [1,5]$, with higher values indicating more critical assessments. The fiscal-tone index is then defined as:
\[
\tau = \frac{3 - \mu}{2}, \qquad \tau \in [-1,1]\,,
\]
a simple rescaling that sets $\tau = 0$ for a neutral document ($\mu = 3$), values near $+1$ for favorable assessments, and values near $-1$ for highly critical ones. This normalization centers the index at zero (neutral tone) and ensures symmetric interpretation: a shift from $\mu=4$ to $\mu=5$ has the same magnitude as a shift from $\mu=2$ to $\mu=1$.\footnote{We verified robustness to alternative aggregation formulas, including the median score (Spearman correlation with $\tau$ = 0.91) and a weighted average emphasizing extreme scores (correlation = 0.96). Main findings are invariant to these choices.}

Panel (b) of Figure~\ref{fig:Results} plots the resulting fiscal--tone series. The index displays short-run fluctuations but exhibits a clear and statistically significant downward trend beginning in 2020 (linear trend coefficient $= -0.012$ per month, $t$-statistic $= -4.18$, $p < 0.001$). During the early years of the sample (2016--2017), the mean tone is $\tau = -0.14$, reflecting moderate concerns but no sustained deterioration. In contrast, the onset of the COVID-19 pandemic produces a sharp and persistent decline to $\tau = -0.47$ (mid-2020), consistent with the surge in severe warnings documented in Panel~(a). Although the tone improves temporarily in 2021 (to $\tau = -0.31$), it remains below zero and never returns to pre-pandemic levels.

The most pronounced fall in the fiscal--tone index occurs from late 2022 onward, a period marked by institutional instability, frequent turnover at the Ministry of Economy and Finance (seven ministers in three years), and a weakening commitment to medium-term fiscal anchors. The index reaches $\tau = -0.52$ (early 2023), levels comparable to (and at times more adverse than) those observed during the height of the pandemic, despite the absence of an external shock. This suggests that the recent deterioration reflects not transitory fiscal pressures but deeper fragilities in fiscal governance and policy implementation. A simple correlation analysis reveals that the fiscal tone index is negatively correlated with the non-financial public sector deficit as a share of GDP ($\rho = -0.64$, $p < 0.01$), indicating that the FC's warnings track actual fiscal outcomes, though with a slight leading pattern (peak correlation at a 3-month lead).


\paragraph{Robustness and limitations.} While the fiscal tone index provides a systematic and replicable measure of FC vigilance, several limitations warrant mention. First, the index captures only \emph{written} communication and does not account for non-public interactions, oral presentations, or informal advice. Second, as with any LLM-based classification, there is a risk of model-specific biases or scoring drift across different model versions, though our validation exercise and deterministic settings mitigate this concern. Third, the index reflects the FC's assessment of fiscal risks, not its actual influence on policy—a critical but separate question requiring analysis of government responses and market reactions. Finally, the corpus is limited to Spanish-language documents from a single country, which may affect generalizability to FCs with different communication styles or institutional mandates.


%---------------------------------------------------
% Conclusions
%---------------------------------------------------

\section{Concluding remarks} \label{section:Conclusions}

This paper has examined the evolution of fiscal discipline in Peru through the lens of the FC's communication and warnings. By applying a large language model to classify the tone of all official publications issued between 2016 and 2025, we constructed a quantitative indicator that tracks how the FC's assessments have shifted in response to economic shocks and institutional change. The results point to a persistent deterioration of fiscal credibility since 2020, driven by weakened rules, expanding legislative discretion, and the erosion of technical counterweights. The fiscal tone index declined by 71\% from the early period (2016--2019, mean $\tau = -0.14$) to recent years (2022--2025, mean $\tau = -0.24$), with the most pronounced deterioration occurring after 2022 despite the absence of exogenous shocks.

More broadly, automated textual analysis of FC documents offers a replicable and quantitative lens through which to observe how institutional signals of fiscal risk evolve over time. This approach bridges computational linguistics and fiscal surveillance, generating early-warning indicators that complement conventional fiscal statistics and strengthen transparency and accountability in the implementation of public finances. Beyond the Peruvian case, the study illustrates the value of analyzing FCs through their behavior rather than only their design. Text-based indicators of vigilance offer a new way to evaluate how these institutions communicate risks, adjust their warnings, and respond to deteriorating fiscal conditions. By showing how LLM-based tools can systematize and quantify this behavior, the paper provides a methodological contribution that can be applied to fiscal councils in other countries, to independent oversight bodies more broadly, and to any institutional setting where communication plays a key monitoring role.

The findings carry important policy implications. First, they suggest that making fiscal tone indices publicly available—updated in real-time as FC documents are released—could enhance transparency and allow markets, media, and civil society to monitor fiscal governance more effectively. Second, the persistent divergence between FC warnings and policy outcomes after 2022 highlights the limits of purely advisory oversight. Institutional reforms that strengthen government accountability to FC assessments—such as mandatory legislative responses to severe warnings, ex-ante fiscal impact requirements for spending initiatives, or budget rules that automatically trigger when the FC tone crosses critical thresholds—may be necessary to restore the credibility of Peru's fiscal framework and ensure that fiscal councils function as genuine mechanisms of fiscal responsibility rather than mere consultative bodies.

\vspace{10mm}

\paragraph{Data Availability Statement.} All Fiscal Council documents analyzed in this study are publicly available at \url{https://cf.gob.pe/p/documentos}. The classified paragraph-level dataset, replication code (Python scripts for PDF processing, LLM classification, and visualization), and the complete LLM prompt will be made available at the authors' institutional repository (\url{https://github.com/[repository]}) upon publication. The GPT-4o API is accessible at \url{https://platform.openai.com} (OpenAI account required). Researchers interested in replicating the analysis or applying the methodology to other fiscal councils are encouraged to contact the corresponding author.

\small
\bibliographystyle{apalike}
\bibliography{COW_Biblio}
\normalsize


%%%%%%%%%%%%%%%%%%%%%%%%%%%%%%%%%%%%%%%%%%%%%%%%%%%%%%%%
% Figures and Tables
%%%%%%%%%%%%%%%%%%%%%%%%%%%%%%%%%%%%%%%%%%%%%%%%%%%%%%%%

\appendix
\clearpage


\begin{figure}[p]
\begin{center}\caption{Fiscal balance, targets and performance in Peru, 2000-2025}\label{fig:Motivation}
\includegraphics[width=0.95\textwidth]{Fig_Performace.png}
\end{center}

\footnotesize  \textbf{Source:} \citet{Montoya2024} and \href{https://estadisticas.bcrp.gob.pe/estadisticas/series/}{BCRPData}. Own elaboration \\[1mm]
\textbf{Notes:} The non-financial public sector balance is shown as a percentage of GDP. Blue lines indicate the annual fiscal target. Light blue bars correspond to years when the rule was met, pink bars to years when it was suspended, and red bars to years of non-compliance. Red "$\times$" marks denote years when the expenditure rule was suspended or not met. The 2020 fiscal deficit (8\% of GDP) is omitted for visual clarity.
\end{figure}

\begin{figure}[p]
    \begin{center}
    \caption{Fiscal warnings and fiscal tone in Peru, 2016-2025}\label{fig:Results} \small

(a) Distribution of scores
\includegraphics[width=0.95\textwidth]{Fig_Distribucion_Context.png} \\[1mm]
 (b) Fiscal tone index \\
\includegraphics[width=0.95\textwidth]{Fig_Tono_Context.png}
\end{center}
\footnotesize \textbf{Source:} Fiscal Council reports and communiqués
(\href{https://cf.gob.pe/p/documentos}{https://cf.gob.pe/p/documentos}). Own elaboration.\\[1mm]
\textbf{Notes:} Panel (a) shows the evolution of the estimated probabilities that paragraphs in each document are classified into one of five alert levels, displayed as a stacked area chart. Panel (b) displays the fiscal tone index $\tau \in [-1,+1]$, where $-1$ indicates maximum fiscal concern and $+1$ indicates a favorable assessment. The dashed line shows the raw monthly series (with linear interpolation for missing months), while the solid line shows a centered moving average for easier visualization. Vertical gray bands mark key events: 2017 El Niño (light), COVID-19 pandemic (medium), and post-2022 institutional crisis (dark). Statistical tests use actual document-level data without interpolation.

\end{figure}

\clearpage
\section{LLM Prompt for Fiscal Tone Classification}

\begingroup
\small
\ttfamily
\noindent
\textbf{[Contextual Framing]}\\[2mm]

Since approximately 2016, the management of public finances in Peru has shown increasing signs of deterioration. The loss of fiscal discipline, lack of transparency, and relaxation of fiscal rules have been recurrent themes in the reports of the Peruvian Fiscal Council. Added to this is the impact of political instability (for example, frequent ministerial changes) on institutional capacity to conduct a prudent and sustainable fiscal policy. In this context, the Fiscal Council has issued increasingly frequent and forceful warnings about the non-fulfillment of fiscal targets, the deterioration of public balances, and the risks of growing and potentially unsustainable indebtedness.

\vspace{3mm}
\noindent
\textbf{[Taxonomy of Fiscal Risk Terminology]}\\[2mm]

Common criteria and terms in the Fiscal Council's reports, organized by category:

\begin{enumerate}
  \item  \textbf{Compliance and fiscal discipline:}   non-fulfillment of fiscal targets (incumplimiento de metas fiscales), relaxation of fiscal rules (relajamiento de reglas fiscales), improper use of public spending (uso inadecuado del gasto público), deviation of the fiscal deficit (desviación del déficit fiscal), deterioration of the fiscal framework (deterioro del marco fiscal), unjustified flexibility (flexibilización sin justificación), procyclical fiscal policy (política fiscal procíclica).

  \item  \textbf{Risk and sustainability:} fiscal risk (riesgo fiscal), debt sustainability risk (riesgo de sostenibilidad de la deuda), excessive indebtedness (endeudamiento excesivo), dependence on extraordinary revenues (dependencia de ingresos extraordinarios), structural fiscal vulnerability (vulnerabilidad fiscal estructural), use of temporary or non-permanent measures (uso de medidas transitorias o no permanentes), macro-fiscal uncertainty (incertidumbre macrofiscal).

  \item \textbf{Governance and institutional capacity:}  fiscal transparency (transparencia fiscal), quality of public spending (calidad del gasto público), institutional uncertainty (incertidumbre institucional), lack of multiannual planning (falta de planificación multianual), frequent changes in economic authorities (cambios frecuentes en autoridades económicas), institutional weakening (debilitamiento institucional), compromised fiscal independence (independencia fiscal comprometida), absence of structural reform (ausencia de reforma estructural).
\end{enumerate}

\vspace{3mm}
\noindent
\textbf{[Scoring Instructions]}\\[2mm]

You are a technical analyst at the Fiscal Council. Evaluate the following paragraph extracted from a technical report of the Fiscal Council, where an opinion is given on the performance of the Ministry of Economy and Finance with respect to the above criteria. Your task is to assign a score from 1 to 5 according to the level of fiscal concern or alert expressed in the text:

\begin{enumerate}
  \item \textbf{1 = No concern:} target compliance, fiscal consolidation, transparency, multiannual planning, institutional strengthening
  \item \textbf{2 = Mild concern:} potential fiscal risk, minor deficit deviation, dependence on extraordinary revenues, temporary measures
  \item \textbf{3 = Neutral:} technical description, management within the framework, no evaluative judgment, purely informational
  \item \textbf{4 = High concern:} target non-compliance, fiscal relaxation, macroeconomic uncertainty, rule violations, institutional weakening
  \item \textbf{5 = Fiscal alarm:} severe criticism, debt sustainability risk, compromised fiscal independence, structural deterioration, crisis warnings
\end{enumerate}

\vspace{2mm}
\noindent
\textbf{Important:} Respond with ONLY a single number (1, 2, 3, 4, or 5). Do not include any explanation or additional text.

\endgroup


\end{document}
